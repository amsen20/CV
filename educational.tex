%%%%%%%%%%%%%%%%%%%%%%%%%%%%%%%%%%%%%%%%%
% Medium Length Professional CV
% LaTeX Template
% Version 2.0 (8/5/13)
%
% This template has been downloaded from:
% http://www.LaTeXTemplates.com
%
% Original author:
% Trey Hunner (http://www.treyhunner.com/)
%
% Important note:
% This template requires the resume.cls file to be in the same directory as the
% .tex file. The resume.cls file provides the resume style used for structuring the
% document.
%
%%%%%%%%%%%%%%%%%%%%%%%%%%%%%%%%%%%%%%%%%

%----------------------------------------------------------------------------------------
%	PACKAGES AND OTHER DOCUMENT CONFIGURATIONS
%----------------------------------------------------------------------------------------

\documentclass{resume} % Use the custom resume.cls style

\usepackage{hyperref}
\usepackage{enumitem}
\usepackage[left=0.5in,top=0.6in,right=0.5in,bottom=0.6in]{geometry} % Document margins
\newcommand{\tab}[1]{\hspace{.2667\textwidth}\rlap{#1}}
\newcommand{\itab}[1]{\hspace{0em}\rlap{#1}}
% \address{4th floor, No. 11, Keinejad Alley, Keynejad St., Shria'ti St., Tehran, Iran}
\address{ \href{https://github.com/amsen20}{GitHub}}
\address{ \href{https://www.linkedin.com/in/amirhossein-pashaee-a53a44b0/}{LinkedIn}}
\address{ahph1380@gmail.com}
\name{AmirHossein PashaeeHir} % Your name


\begin{document}

%----------------------------------------------------------------------------------------
%	EDUCATION SECTION
%----------------------------------------------------------------------------------------

\begin{rSection}{Education}

    {\bf Amirkabir University of Technology} \hfill {\em Sept 2018 - Present} 
    \\ 4th Year, BS Degree \hfill { Overall GPA: 18.89/20}
    \\ Department of Computer Engineering
    \vspace{2mm}\\
    {\bf Allame Helli 1 High School} \hfill {\em Sept 2014 - Sept 2018} 
    \\ Diploma of Theoretical Mathematics and Physics


\end{rSection}
%----------------------------------------------------------------------------------------
%	TECHNICAL STRENGTHS SECTION
%----------------------------------------------------------------------------------------
\begin{rSection}{Research Experience}

    {\bf Research Assistant at Amirkabir University of Technology} \hfill {\em Sept 2021 - Present}\\
    {\em Under supervision of M. Momtazpour}
    \vspace{2mm}\\
    Working on designing a scheduler for mixed edge and cloud Kubernetes clusters. The proposed scheduler aims to minimize resource fragmentation in the edge cluster to decrease average response time by increasing resource utilization on the edge.
\end{rSection}

\begin{rSection}{Professional Experience}
	{\bf Full Stack Developer at a hedge fund in Singapore} \hfill {\em June 2022 - Present}\\
	Contributing to developing a package manager for C++ using Django for the back-end and Bazel for client-side library management.
	\vspace{2mm}\\
	Developing a light weight failure-robust job scheduler for multi-layer clusters for error aggregation purposes using Go.
	\vspace{2mm}\\
	Developing a concurrent arithmetic expression compiler for processing time series data streams. The parser is implemented in yacc, lex, and Go.
	\vspace{2mm}\\
	{\bf {Full Stack Developer at Quera}} \hfill {\em Aug 2019 - Sept 2020}\\
	Contributed to developing Quera Colleague platform, written in Django Python for the back-end and React, Redux JS for the front-end.
	\vspace{2mm}\\
	{\bf {Advanced Python Course Writer at Quera}} \hfill {\em March 2019 - Aug 2019}\\
	Wrote a Python course from beginner to advanced in Quera Colleague.
	\vspace{2mm}\\
	{\bf {C++ Developer at Fanap Soft}} \hfill {\em Nov 2018 - March 2019}\\
	Contributed to developing \href{https://playpod.ir/}{PlayPod} (first cloud gaming platform in middle east) Windows client, which is written in C++ using ImGui.
\end{rSection}

\begin{rSection}{Honors and Awards} \itemsep -2pt
	{\bf ICPC}, International Collegiate Programming Contest\\
	The International Collegiate Programming Contest is the most prestigious programming contest for college students.
	\begin{itemize}[leftmargin=*, label=\raisebox{0.25ex}{\tiny$\bullet$}]
		\item
		{\bf 24\textsuperscript{th} team} in the \href{https://icpc.global/community/results-2020}{44\textsuperscript{th} ICPC World Finals, Moscow, Russia} 
		\hfill {\em 2021}\\
		Top 0.2\% among more than 45,000 students from all over the world.
		\item
		{\bf 2\textsuperscript{nd}, 4\textsuperscript{th}
			team} in Regional Contests of ICPC West Asia Region, Tehran site, respectively in \href{http://icpc.sharif.edu/2019/scoreboard/}{2019} and \href{http://icpc.sharif.edu/2018/scoreboard/}{2018}.
		Top 1\% among more than 300 teams that participate in this contest every year.
		\item National Countrywide University Entrance Exam(aka Konkour), Iran \hfill {\em 2018}\\
		Rank top 0.1\% in among more than 170,000 participants.
	\end{itemize}
\end{rSection}

\begin{rSection}{Notable Projects}

 
    {\bf \href{https://github.com/amsen20/Sal}{Sal}}
    %\hfill {\em June 2021 - Sep 2021}
    \\
	The Sal compiler compiles simple python programs to Sal files which are executable by the Sal VM. The Sal file describes the python program as a directed acyclic graph (DAG) capable of generating infinite graphs with similar features as computation graphs.
    \vspace{2mm}\\
    {\bf \href{https://github.com/amsen20/hackerearth-problems}{My hackerearth problems}} 
    %\hfill {\em Jan 2021 - Jul 2021}
    \\
    My problems judges (generator, validator, checker, and solution) in \href{https://www.hackerearth.com/}{hackerearth.com} contests. Problems' difficulty varies from easy to hard. the problem set includes graphs, combinatorics, constructive, math, strings, and approximate problems.
    \vspace{2mm}\\
    {\bf NUMEX}
    \\
    The pure functional programming language (NUMEX) interpreter, built with Racket. (Programming languages course project)
    \vspace{2mm}\\
    {\bf \href{https://github.com/amsen20/search-engine}{Search engine}}
    \\
    A search engine for retrieving news documents. Both statistical and word embedding approaches are used for answering users' queries. (Information retrieval course project)
    \vspace{2mm}\\
    {\bf \href{https://github.com/amsen20/MCP/tree/main/project/phase-two-GPU}{GPU-accelerated neural network}} % \hfill {\em March 2022 - April 2022}
    \\
    A GPU-accelerated version of \href{https://www.google.com/search?q=genann&oq=genann&aqs=chrome.0.69i59j69i57j69i59j69i60l5.721j0j4&sourceid=chrome&ie=UTF-8}{genann}. Memory coalescing, vectorize operations, reduction, and other techniques are used in order to distribute the processing over GPU. A 2x speedup is gained on an MX450 Nvidia GPU. (Multicore programming course project)
\end{rSection}
%----------------------------------------------------------------------------------------
%	WORK EXPERIENCE SECTION
%----------------------------------------------------------------------------------------
\begin{rSection}{Teaching Assistant Experience}
	{{\bf Head TA}, Programming Languages, M. S. Fallah} \hfill {\em Fall 2021}\\
    {{\bf Head TA}, Data Structures and Algorithms, H. Hoorfar} \hfill {\em Fall 2020}\\
    {{\bf Head TA}, Algorithm Design, H. Hoorfar} \hfill {\em Spring 2021}\\
    {Electric and Electronic Circuits, M. Momtazpour} \hfill {\em Fall 2021}\\
    {Microprocessor and Assembly, H. Frabeh} \hfill {\em Spring 2022}\\
\end{rSection}

%----------------------------------------------------------------------------------------
%	TECHNICAL STRENGTHS SECTION
%----------------------------------------------------------------------------------------

%----------------------------------------------------------------------------------------

% \begin{rSection}{Academic Service}
%     \begin{itemize}[leftmargin=*]
%     \item
%         {\bf AI Challenge 2020}
%         \\ Developing Java client code for participants
%     \item
%         {\bf Webelopers 2020}
%         \\ Technical Staff
%     \item
%         {\bf ACM 2020}
%         \\ Technical Staff
%     \item
%         {\bf Rasta Summer School 2019}
%         \\ Mentoring the cryptography section

%     \end{itemize}
% \end{rSection}


\begin{rSection}{Relevant Courses}
        {\bf University Major Courses:}
        Program Analysis, Programming Languages, Theory of Machines and Languages, Principles of Compiler Design, Multicore Programming, Principles of Cloud Computing, Computer Networks, Data Mining, Principles of Computational Intelligence, Information Retrieval, Research and Technical Presentation
\end{rSection}

%----------------------------------------------------------------------------------------
%\begin{rSection}{Languages}
%    Persian (Native Proficiency)\\
%    English (Full Professional Proficiency)\\
%\end{rSection}

\begin{rSection}{Skills and Qualities}
    {\bf Programming Languages:}
    Expert in C++, Python. Working knowledge in Go, C, Haskell, Racket, Javascript, Java, SQL, \LaTeX.
    \\
    {\bf Libraries:}
    Expert in STL, Redux. Working knowledge in GLPK, ImGUI, Numpy, Pandas.
    \\
    {\bf Frameworks:} Expert in Django, React JS.
    \\
    {\bf DevOps:} Developed some modules for Kubernetes. Working knowledge in Git, Linux, Bash. 
    \\
    {\bf Software Engineering:} Familiar with object-oriented design patterns, concurrency patterns, functional programming, and procedural programming.
    \\
    {\bf Theoretical Background:} Expert in design of algorithms, computation theory, data structures, and discrete mathematical fields such as
    graph theory and combinatorics suggested by awards in ICPC.
    \\
    {\bf Languages:} English (Full professional proficiency), Persian (Native).
\end{rSection}


%\begin{rSection}{Interests}
%    Automatic programming, Distributed computing, Parallel %programming, Automated reasoning, Algorithms, Functional %programming 
%\end{rSection}

%\begin{rSection}{Hobbies}
%    Paleontology, Music, Basketball, Board games
%\end{rSection}

\end{document}

